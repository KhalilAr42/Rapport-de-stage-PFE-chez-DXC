\newpage

\titleformat % design des titres des chapitres
{\chapter}
[display]
{\centering\normalfont\Large\scshape\bfseries}
{\rule[3pt]{0.15\linewidth}{3pt}\quad\chaptertitlename~\thechapter\quad \rule[3pt] {0.15\linewidth}{3pt}}
{0\baselineskip}%espace vertical entre chapitre et nom du chapitre
{\rule{\linewidth}{0.5pt}\break\Huge}
[\vspace{-0.5\baselineskip}\rule{\linewidth}{0.5pt}\vspace{0\baselineskip}]

\let\clearpage\relax% Stop LaTeX from going to a new page; and
\vspace*{5.5cm}%

\chapter{Etudes des besoins}
Le présent chapitre a pour but de définir les besoins fonctionnels et non
fonctionnels, après avoir décrit les processus métiers, et les règles de gestion.

\newpage

\section{Description du projet}
\subsection{Objectifs fonctionnels du projet:}

L’objectif principal du projet est la facilitation de saisie des données ainsi que l’aide à la prise de décision, de passer d'un modele de travail qui utilisait plusieurs fichier excel partager avec plusieur teams pour prendre la decision et s'organiser vers un model qui est centré sur une seul application Power Apps qui met en relation les differentes entittés du system.
\\

Dans un premier temps, il s’agit de définir de manière claire et précise les besoins et les attentes d’un système d’information permettant d’automatiser les remontées de l’équipe vers différentes entités, ainsi que la saisie des Deals, choix des Squads, l’affectation des ressources et l’affichage des statistiques dans un dashboard avec differents filtre.
\\

Fournir une visibilité sur les prévisions financières au niveau du compte et du portefeuille, y compris les changements de revenus mensuels et trimestriels. Ce système devra être capable de gérer les relations clients, la saisie des différents deals pour assurer un suivi par les Services Lines, un Reporting et un suivi des indicateurs.
\\

Dans un second temps, le développement du système d’information devra permettre
de garder la trace de tous les échanges afin de pouvoir établir des statistiques et prendre la décision du gain du deal. 
\\

Tout au long du projet, la notion de passage à l’échelle devra être prise en compte. L’objectif à long terme de la conception et du développement d’un tel système d’information est de pouvoir être utilisé par tous les acteurs de l’entreprise.

\section{Contraintes du projet}

\subsection{Contraintes en termes de délais}
A partir de la livraison du cahier des charges, nous disposons d’environ quatre mois pour la réalisation du projet. Le délai semble court mais reste suffisant pour se concentrer sur la partie prévue pour le projet de fin d’études.

\subsection{Contraintes de sécurité}
La gestion de la sécurité est la principale contrainte de notre système. L'application doit posséder une gestion de privilèges et de niveaux d'accès pour les différents types d'utilisateurs (RH, administration, ...). Selon leur statut, le contenu des pages varie et l'accès aux informations avec un statut supérieur est interdit.

\subsection{Contraintes techniques}

Pour le développement de notre système, nous disposons d’une architecture
existante sur laquelle nous devrons baser notre application. La structure de notre système doit être extensible pour la mettre en place facilement dans les autres unités de l’entreprise. De plus, le développement devra suivre toutes les normes techniques pour une meilleure performance, maintenance et facilité de mise à jour.